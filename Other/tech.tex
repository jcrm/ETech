\subsection{Technology: }\label{tech}
When creating the application it was designed and built upon the main aspects of augmented reality applications, the use of a marker as an anchor in the 3D world.
This application builds all the objects around a set of designated markers.
Figure \ref{sc:position_pseudo} shows the pseudo code that does this process.

\begin{figure}[h|]
\centering
\begin{lstlisting}
for num of markers{
	if seen{
		set cube ar matrix();
	}else{
		clear ar matrix;
	}
}

set cube ar matrix(){
	for num of faces{
		set puzzle ar matrix();
	}
}

set puzzle ar matrix(){
	set ar matrix;
	build new world transform matrix;
}
\end{lstlisting}
\caption{Pseudo Code}
\label{sc:position_pseudo}
\end{figure}

In Figure \ref{sc:position_pseudo} the augmented reality marker location matrix is sent to the main cube object.
From here there matrix is then passed to the faces of the cube to be used by the object as part of its own world matrix.

\begin{figure}[h|]
\centering
\begin{lstlisting}
for each tile{
	//local rotation multiplied by local position for the puzzle
	build local martrix();
	//get the tile matrix and multiply by local matrix;
	tile matrix * local matrix;
	current matrix * ar matrix;
	set the world matrix of each tile to be the current matrix;
}
\end{lstlisting}
\caption{Matrix Pseudo Code}
\label{sc:matrix_pseudo}
\end{figure}

An example of how the world matrix is used by a tile in a puzzle can be seen in Figure \ref{sc:matrix_pseudo}.
This code shows that each tile starts by first creating its own world matrix based upon its own rotation and position.
Next this matrix is the multiplied by the matrix of the cube object it is relative to.
This results in the position of the tile in the world being placed just above one of the faces of the cube.
The next step is what sets the tile relative to the augmented reality marker, which is by multiplying the current matrix by the augmented reality matrix returned from the augmented reality marker.

As well as the application development working well with the exploits of augmented reality the gameplay of the application works well with how augmented reality works.
As mentioned previously when playing the game the user has to solve the puzzle on each face.
The best way to do this is by moving around the cube while changing the face needing to be solved at the time.
This process allows the user to see each face of the cube without interrupting the gameplay to move the marker, which may not be possible if the marker is in a fixed location.
