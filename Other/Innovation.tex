\subsection{Innovation:}\label{sc:innov}

The previous section talked about how the application evaluates compared to the start of the project and how it could improve to be a better version.
This section will look at how the application could use innovation of the augmented reality to further the game. 

The way the current game works the user has to find a marker and solve the puzzles on the cube.
This process can be put slightly altered to extend the functionality of the application. For example the markers could be set up at an event such as a festival, once a user has found all the markers they could redeem a voucher for free item either digitally or physically.
This could lead to events such as \cite{munzee11}. \cite{munzee11} is a global scavenger hunt in which the user has to scan a QR code. This QR code be a marker for an augmented reality object which the user has to solve before continuing the scavenger hunt.

An innovation in augmented reality are devices such as CastAR, \citealp{castar}, which instead of looking at the augmented reality through a small device screen the user wears glasses which displays the augmented world.
The CastAR glasses with the use of the Wand joystick and 3D input device could be used with the application in the future to allow for more immersive gameplay.
Such as giant puzzles appearing in the street in front of the user.